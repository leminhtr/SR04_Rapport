\chapter{Conclusion}\label{ch:conclusion}

Ce projet expérimental dans le cadre de l'UV SR04 fut riche en apprentissage de nouvelles connaissances théoriques et pratiques pour l'ensemble de notre groupe de projet. 

Nous avons pu revoir en détail la théorie du modèle OSI (Open Systems Interconnection) et le fonctionnement de sa couche réseau (couche 3). Nous avons également parlé des spécificités du routage et avons étudié en détail différents protocoles de la couche réseau :  l'IPv6, l'ICMPv6. Nous avons su montrer quelles sont les nouveautés et avantages apportés par la norme IPv6 par rapport à la norme IPv4.

Nous avons ensuite présenté dans notre rapport les grands principes et mises en œuvre de la Qualité de Service (QoS) qui était le sujet central de notre étude.

Une fois l'étude théorique bien avancée, nous avons pu commencer à tester en pratique les principes étudiés précédemment en théorie. Il nous a fallu pour cela apprendre à configurer et manipuler les fonctionnalités de base d'un routeur Cisco. Puis nous avons mis en fonctionnement et testé le routage, l'adressage, l'autoconfiguration en IPv6 sur ce routeur.

Finalement, nous avons mis en place des configurations du routeur avec et sans QoS. Puis nous avons tenté de saturer le routeur en créant du trafic à l'aide de plusieurs machines et plusieurs applications (conférence audio, transfert de fichier, téléphonie, streaming vidéo) le tout de manière simultanée.

Ces tests expérimentaux, mettant en place la QoS pour les services de VoIP, nous ont permis d'apprendre à installer une QoS simple dans un réseau et à démontrer l'efficacité de la QoS dans le but de favoriser certaines applications dans une architecture réseau.

Nous tenons particulièrement à remercier M. Bouabdallah et l'équipe pédagogique de l'UV SR04 pour leur soutien, leur aide et leurs conseils apportés tout au long du projet.

